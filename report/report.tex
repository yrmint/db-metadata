\documentclass[12pt]{article}
\usepackage[russian]{babel}
\usepackage{indentfirst}
\usepackage{graphicx}
\usepackage [left=30 mm, top=15 mm, right=30 mm, bottom=20mm, nohead, footskip=10 mm] {geometry}

\parindent=24pt

\begin{document}
	\thispagestyle{empty}

\begin{center}
	\large{МИНОБРНАУКИ РОССИИ} \par
	\vspace{0.3cm}
	\normalsize
	{ФЕДЕРАЛЬНОЕ ГОСУДАРСТВЕННОЕ АВТОНОМНОЕ ОБРАЗОВАТЕЛЬНОЕ УЧРЕЖДЕНИЕ ВЫСШЕГО ОБРАЗОВАНИЯ} \par
	\vspace{0.3cm}
	\textbf{\guillemotleft САНКТ-ПЕТЕРБУРГСКИЙ ПОЛИТЕХНИЧЕСКИЙ}
	\textbf{УНИВЕРСИТЕТ ПЕТРА ВЕЛИКОГО\guillemotright} \par
	\vspace{0.3cm}
	
	{Институт компьютерных наук и кибербезопасности}\par
	{Высшая школа технологий искусственного интеллекта}\par
	{Направление 02.03.01 Математика и Компьютерные науки}	
\end{center}

\vfill

\begin{center}
	{\LARGE КУРСОВАЯ РАБОТА} \par
	\vspace{0.3cm}
	{\large по дисциплине \guillemotleft Управление знаниями и технологии баз данных\guillemotright}\par
	{\LARGE Разработка и программирование базы метаданных\\ }\par
\end{center}

\vfill

\begin{flushleft}
	Студент: \hspace{1.8cm} \rule[0pt]{2.5cm}{0.5pt}\hfill Жилкина Лада Михайловна\par
	\vspace{1.5cm}
	Преподаватель: \hspace{0.55cm} \rule[0pt]{2.5cm}{0.5pt}\hfill  Попов Сергей Геннадьевич
\end{flushleft}

\vspace{0.5cm}

\begin{flushright}
	\guillemotleft \rule[0pt]{0.8cm}{0.5pt}\guillemotright \rule[0pt]{2cm}{0.5pt} 20\rule[0pt]{0.5cm}{0.5pt} г.
\end{flushright}

\vfill

\begin{center}
	Санкт-Петербург -- 2025
\end{center}	
	
	
	\thispagestyle{empty}
	\newpage
	\tableofcontents
	
	
	
	\newpage
	\section{Постановка задачи}
	
	\par В рамках курсовой работы необходимо разработать программное решение, направленное на автоматизированное извлечение, хранение и управление метаданными баз данных.
	
	\begin{enumerate}
		\item Предусмотреть возможность подключения и работы с несколькими базами данных MySQL, содержащими произвольные структуры таблиц и данных.
		\item Разработать и реализовать схему базы данных для хранения метаданных, включающую информацию о:
		\begin{itemize}
			\item серверах баз данных,
			\item самих базах данных,
			\item таблицах,
			\item сущностях,
			\item ключах.
		\end{itemize}
		\item Реализовать механизм автоматизированной выгрузки метаданных из подключаемых баз данных и сохранения их в разработанную базу.
		\item Реализовать интерфейс, позволяющий пользователю получать статистическую информацию на основе базы метаданных, такую как:
		\begin{itemize}
			\item общее количество подключённых баз данных,
			\item количество таблиц в каждой базе,
			\item количество столбцов, ключей и других структурных элементов,
			\item другие показатели, связанные со структурой и организацией данных.
		\end{itemize}
	\end{enumerate}

	\newpage
	\section{Схема хранилища}
	\subsection{Таблицы базы данных}
	
	\par{\bfБаза данных}
	\begin{table}[h]
		\centering
		\begin{tabular}{|c|c|c|c|c|}
			\hline
			№ & Название & Тип & Тип ключа & Ссылка \\
			\hline
			1 & db\_id & INT & PK & - \\
			\hline
			2 & db\_name & VARCHAR(64) & UNIQUE & - \\
			\hline
		\end{tabular}
		\caption{Сущность База данных (Db)}
	\end{table}
	
	\par{\bfТаблица}
	\begin{table}[h]
		\centering
		\begin{tabular}{|c|c|c|c|c|}
			\hline
			№ & Название & Тип & Тип ключа & Ссылка \\
			\hline
			1 & table\_id & INT & PK & - \\
			\hline
			2 & table\_name & VARCHAR(64) & - & - \\
			\hline
			3 & db\_id & INT & FK & Db(db\_id) \\
			\hline
		\end{tabular}
		\caption{Сущность Таблица (Db\_table)}
	\end{table}
	
	\par{\bfКолонка}
	\begin{table}[h]
		\centering
		\begin{tabular}{|c|c|c|c|c|}
			\hline
			№ & Название & Тип & Тип ключа & Ссылка \\
			\hline
			1 & column\_id & INT & PK & - \\
			\hline
			2 & column\_name & VARCHAR(64) & - & - \\
			\hline
			3 & table\_id & INT & FK & Db\_table(table\_id) \\
			\hline
		\end{tabular}
		\caption{Сущность Колонка (Db\_column)}
	\end{table}
	
	\newpage
	\par{\bfКлюч}
	\begin{table}[h]
		\centering
		\begin{tabular}{|c|c|c|c|c|}
			\hline
			№ & Название & Тип & Тип ключа & Ссылка \\
			\hline
			1 & constraint\_id & INT & PK & - \\
			\hline
			2 & constraint\_name & VARCHAR(64) & - & - \\
			\hline
			3 & type & ENUM & - & - \\
			\hline
			4 & column\_id & INT & FK & Column(column\_id) \\
			\hline
		\end{tabular}
		\caption{Сущность Ключ (Constraint)}
	\end{table}
	
	\par{\bfКлюч-колонка}
	\begin{table}[h]
		\centering
		\begin{tabular}{|c|c|c|c|c|}
			\hline
			№ & Название & Тип & Тип ключа & Ссылка \\
			\hline
			1 & constraint\_column\_id & INT & PK & - \\
			\hline
			2 & position & INT & - & - \\
			\hline
			3 & constraint\_id & INT & FK & Constraint(constraint\_id) \\
			\hline
			4 & column\_id & INT & FK & Column(column\_id) \\
			\hline
		\end{tabular}
		\caption{Сущность Ключ-колонка (Constraint\_column)}
	\end{table}
	
	\par{\bfВнешний ключ}
	\begin{table}[h]
		\centering
		\begin{tabular}{|c|c|c|c|c|}
			\hline
			№ & Название & Тип & Тип ключа & Ссылка \\
			\hline
			1 & rc\_id & INT & PK & - \\
			\hline
			2 & fk\_constraint\_id & INT & FK & Constraint(constraint\_id) \\
			\hline
			3 & pk\_constraint\_id & INT & FK & Constraint(constraint\_id) \\
			\hline
		\end{tabular}
		\caption{Сущность Внешний ключ (Referential\_constraint)}
	\end{table}
	
	\newpage
	\begin{figure}
 	 	\includegraphics[width=\linewidth]{db_structure}
  		\caption{Структурная схема хранилища метаданных.}
  		\label{fig:db_structure}
	\end{figure}
	

	
\end{document}